% !TeX program = lualatex
%----------------------------------------------------------------------------------------
%  PACKAGES AND OTHER DOCUMENT CONFIGURATIONS
%----------------------------------------------------------------------------------------

\documentclass[20pt,final]{beamer}

\usepackage[scale=1]{beamerposter} % Use the beamerposter package for laying out the poster
\usepackage[utf8]{inputenc}
\usepackage[english]{babel}

\usetheme{confposter} % Use the confposter theme supplied with this template

\setbeamercolor{titlelike}         {bg=yaleblue,fg=white}
\setbeamercolor{frametitle}        {bg=yaleblue!10,fg=yaleblue}
\setbeamercolor{block title}{fg=yaleblue,bg=white} % Colors of the block titles
\setbeamercolor{block body}{fg=black,bg=white} % Colors of the body of blocks
% \setbeamercolor{block alerted title}{fg=white,bg=dblue!70} % Colors of the highlighted block titles
% \setbeamercolor{block alerted body}{fg=black,bg=dblue!10} % Colors of the body of highlighted blocks
\setbeamercolor{block alerted title}{fg=white,bg=calyellow}
\setbeamercolor{block alerted body}{fg=black,bg=white}

\setbeamercolor{enumerate item}{fg=dblue!70}
\setbeamercolor{itemize item}{fg=dblue!70}
\setbeamercolor{itemize subitem}{fg=dblue!70}


%-----------------------------------------------------------
% Define the column widths and overall poster size
% To set effective sepwid, onecolwid and twocolwid values, first choose how many columns you want and how much separation you want between columns
% In this template, the separation width chosen is 0.024 of the paper width and a 4-column layout
% onecolwid should therefore be (1-(# of columns+1)*sepwid)/# of columns e.g. (1-(4+1)*0.024)/4 = 0.22
% Set twocolwid to be (2*onecolwid)+sepwid = 0.464
% Set threecolwid to be (3*onecolwid)+2*sepwid = 0.708

\newlength{\sepwid}
\newlength{\onecolwid}
\newlength{\twocolwid}
\newlength{\threecolwid}
\setlength{\paperwidth}{42in} % A0 width: 46.8in
\setlength{\paperheight}{33.5in} % A0 height: 33.1in
\setlength{\sepwid}{0.02\paperwidth} % Separation width (white space) between columns
\setlength{\onecolwid}{0.3\paperwidth} % Width of one column
\setlength{\twocolwid}{2\onecolwid} % Width of two columns
\addtolength{\twocolwid}{\sepwid}
\setlength{\threecolwid}{3\onecolwid} % Width of three columns
\addtolength{\threecolwid}{2\sepwid}
\setlength{\topmargin}{-0.5in} % Reduce the top margin size
\setlength\labelsep{\dimexpr\labelsep + 0.5em\relax}
\setlength{\parskip}{1em}
\setlength\columnsep{\sepwid}

\setbeamertemplate{itemize/enumerate body begin}{\normalsize}
\setbeamertemplate{itemize/enumerate subbody begin}{\normalsize}
\setbeamertemplate{itemize/enumerate subsubbody begin}{\normalsize}

% declare `cmex` to be arbitrary scalable
\DeclareFontShape{OMX}{cmex}{m}{n}{
  <-7.5> cmex7
  <7.5-8.5> cmex8
  <8.5-9.5> cmex9
  <9.5-> cmex10
}{}

\SetSymbolFont{largesymbols}{normal}{OMX}{cmex}{m}{n}
\SetSymbolFont{largesymbols}{bold}  {OMX}{cmex}{m}{n}

%-----------------------------------------------------------
% \renewcommand{\figurename}{Fig.}

\usepackage{multicol}
\setbeamerfont{caption}{size=\normalsize}
\usepackage{caption}
\captionsetup{labelformat=simple, labelsep=colon}

\usepackage{graphicx}  % Required for including images

\usepackage{booktabs} % Top and bottom rules for tables

\usepackage{bbm} % more BB
\input{\string~/HeadRs/common_supplement.tex}

\usepackage[sorting=none]{biblatex}
\renewcommand*{\bibfont}{\footnotesize}
\addbibresource{ser_poster.bib}

% ---------------------------------------------------------------------------
\setbeamertemplate{headline}{
 \leavevmode
  \begin{columns}
   \begin{column}{\linewidth} % line
    \vskip1cm
    \centering
    \usebeamercolor{title in headline}{\color{jblue}
        \Huge % CHANGE this line for the size of the title
        {\textbf{\inserttitle}}\\[0.5ex]}
    \usebeamercolor{author in headline}{\color{fg}
        \Large % CHANGE this line for the size of the author
        {\insertauthor}\\[1ex]}
    \usebeamercolor{institute in headline}{\color{fg}
        \large % CHANGE this line for the size of the institut
        {\insertinstitute}\\[1ex]}
    \vskip1cm
   \end{column}
   \vspace{1cm}
  \end{columns}
 \vspace{0.5in} % CHANGE this line for the space between your title and the horizontal rule
 \begin{beamercolorbox}[wd=47in,colsep=0.1cm]{cboxb}\end{beamercolorbox}
 \vspace{0.1in} % CHANGE this line for the space between your horizontal rule and your main body
}

%----------------------------------------------------------------------------------------
%  TITLE SECTION
%----------------------------------------------------------------------------------------
\title{\huge Risk of non-Hodgkin lymphoma under hypothetical interventions with guaranteed positivity} % Poster title
\author{Kevin T. Chen\textsuperscript{\small 1}, Sally Picciotto\textsuperscript{\small 2}, Ellen A. Eisen\textsuperscript{\small 2}} % Author(s)
\institute{University of California, Berkeley -- School of Public Health} % Institution(s)
%----------------------------------------------------------------------------------------

\begin{document}

% add logo
\addtobeamertemplate{headline}{}
{\begin{tikzpicture}[remember picture, overlay]
    \node [anchor=north west, inner sep=1.5cm]  at (current page.north west)
    {\includegraphics[height=9cm]{ucberkeleyseal_139_540.eps}};
  \end{tikzpicture}}
\addtobeamertemplate{block end}{}{\vspace*{2ex}} % White space under blocks
\addtobeamertemplate{block alerted end}{}{\vspace*{2ex}} % White space under highlighted (alert) blocks

\setlength{\belowcaptionskip}{2ex} % White space under figures
\setlength\belowdisplayshortskip{2ex} % White space under equations

\begin{frame}[t] % The whole poster is enclosed in one beamer frame

  \vspace{0.01\paperheight}

  \begin{columns}[t,totalwidth=\onecolwid]

    \begin{column}{\sepwid}\end{column} % Empty spacer column

    \begin{column}{\onecolwid} % The first column
      \vbox to 0.78\paperheight{

        %----------------------------------------------------------------------------------------
        %  OBJECTIVES
        %----------------------------------------------------------------------------------------

        \begin{block}{Backgound}

          \begin{itemize}
            \item \textbf{Non-Hodgkin Lymphoma} (NHL) incidence was associated with
                  exposure to
                  \textbf{soluble metalworking fluid} (MWF) in a
                  Cox analysis of the
                  \textbf{United Auto Workers-General Motors} (UAW-GM) \textbf{cohort}
                  \nocite{Colbeth_2022}
            \item Unlike traditional regression analysis, \emph{causal inference} methods
                  \begin{itemize}
                    \item Can adjust for time-varying confounding affected by past exposure \nocite{Arrighi_1994}
                    \item Estimate population effects of hypothetical interventions
                  \end{itemize}
            \item Causal inference in statistics requires \textbf{positivity}, which is
                  \textbf{not always assessed or addressed} \nocite{Petersen_2012}
            \item Here, we specified \textbf{\emph{supportable} interventions} on soluble MWF exposure
                  in the UAW-GM cohort that \textbf{guarantee positivity}
            \item We estimated the effect of supportable interventions on
                  NHL risk in the UAW-GM cohort (1985-2015) using the
                  \textbf{hazard-extended iterative conditional expectation (ICE) parametric g-formula}
                  \nocite{Wen_2020,Young_2014}
            \item Unlike the classic parametric g-formula, ICE g-formula estimators do not require
                  parametric specification of the full joint distribution of the confounders,
                  exposure, and outcome
          \end{itemize}

        \end{block}


        \begin{block}{Target and supportable exposure limits}

          \begin{itemize}
            \item Suppose we want to know the effect of capping exposures at a \textbf{target exposure limit},
                  but estimation may not be supported by observed data
            \item Instead, we define \textbf{supportable exposure limits}
                  for every time $k$ and unique combination of confounder and exposure histories
                  $(\bar l_{k}, \bar a_{k - 1})$
                  \begin{itemize}
                    \item Limit at greatest observed exposure $\le$ \textbf{target limit}, if exists
                    \item No limit, if all observed exposures $>$ \textbf{target limit}
                  \end{itemize}
            \item Propensity scores for exposure at the supportable exposure limits
                  are guaranteed to be \textbf{strictly positive}
                  % \item Supportable exposure limits may be equal to, less than, or greater than the target exposure limits
            \item \textbf{Supportable intervention rule} for every
                  $(\bar l_{k}, \bar a_{k - 1})$, reduces exposures $a_k$ above the
                  \textbf{supportable exposure limit} to that limit, but allows exposures
                  below to vary naturally (Figures 1 and 2)
            \item Applying the supportable intervention rule to the observed data
                  induces the intervention distribution, which defines the
                  \textbf{stochastic dynamic intervention} with \textbf{guaranteed positvity}
          \end{itemize}

          \vspace{0.1\baselineskip}
          \begin{figure}
            \caption{Marginal distribution of nonzero exposure at $k = 2$
              before and after applying the supportable intervention rule}
            \begin{tikzpicture}[x=2.5pt,y=2.5pt]
              % Created by tikzDevice version 0.12.3.1 on 2023-06-08 17:09:01
% !TEX encoding = UTF-8 Unicode
\definecolor{fillColor}{RGB}{255,255,255}
\path[use as bounding box,fill=fillColor,fill opacity=0.00] (0,0) rectangle (271.01,195.13);
\begin{scope}
\path[clip] (  0.00,  0.00) rectangle (271.01,195.13);
\definecolor{drawColor}{RGB}{255,255,255}
\definecolor{fillColor}{RGB}{255,255,255}

\path[draw=drawColor,line width= 0.6pt,line join=round,line cap=round,fill=fillColor] ( -0.00,  0.00) rectangle (271.01,195.13);
\end{scope}
\begin{scope}
\path[clip] ( 36.40, 72.85) rectangle (265.01,195.13);
\definecolor{fillColor}{RGB}{255,255,255}

\path[fill=fillColor] ( 36.40, 72.85) rectangle (265.01,195.13);
\definecolor{fillColor}{RGB}{163,207,235}

\path[fill=fillColor] ( 46.79, 78.40) rectangle ( 58.33, 78.43);

\path[fill=fillColor] ( 58.33, 78.40) rectangle ( 69.88, 78.79);

\path[fill=fillColor] ( 69.88, 78.40) rectangle ( 81.43, 78.79);

\path[fill=fillColor] ( 81.43, 78.40) rectangle ( 92.97, 79.38);

\path[fill=fillColor] ( 92.97, 78.40) rectangle (104.52, 79.86);

\path[fill=fillColor] (104.52, 78.40) rectangle (116.06, 80.51);

\path[fill=fillColor] (116.06, 78.40) rectangle (127.61, 83.35);

\path[fill=fillColor] (127.61, 78.40) rectangle (139.16, 85.54);

\path[fill=fillColor] (139.16, 78.40) rectangle (150.70,103.29);

\path[fill=fillColor] (150.70, 78.40) rectangle (162.25,102.34);

\path[fill=fillColor] (162.25, 78.40) rectangle (173.80,189.57);

\path[fill=fillColor] (173.80, 78.40) rectangle (185.34,123.53);

\path[fill=fillColor] (185.34, 78.40) rectangle (196.89,106.28);

\path[fill=fillColor] (196.89, 78.40) rectangle (208.44, 88.77);

\path[fill=fillColor] (208.44, 78.40) rectangle (219.98, 85.04);

\path[fill=fillColor] (219.98, 78.40) rectangle (231.53, 78.97);

\path[fill=fillColor] (231.53, 78.40) rectangle (243.07, 79.17);

\path[fill=fillColor] (243.07, 78.40) rectangle (254.62, 78.43);
\definecolor{drawColor}{RGB}{186,129,35}

\path[draw=drawColor,line width= 3.2pt] ( 46.79, 78.40) rectangle ( 58.33, 78.43);

\path[draw=drawColor,line width= 3.2pt] ( 58.33, 78.40) rectangle ( 69.88, 78.73);

\path[draw=drawColor,line width= 3.2pt] ( 69.88, 78.40) rectangle ( 81.43, 78.76);

\path[draw=drawColor,line width= 3.2pt] ( 81.43, 78.40) rectangle ( 92.97, 79.20);

\path[draw=drawColor,line width= 3.2pt] ( 92.97, 78.40) rectangle (104.52, 79.47);

\path[draw=drawColor,line width= 3.2pt] (104.52, 78.40) rectangle (116.06, 79.85);

\path[draw=drawColor,line width= 3.2pt] (116.06, 78.40) rectangle (127.61, 82.26);

\path[draw=drawColor,line width= 3.2pt] (127.61, 78.40) rectangle (139.16, 83.65);

\path[draw=drawColor,line width= 3.2pt] (139.16, 78.40) rectangle (150.70, 93.74);

\path[draw=drawColor,line width= 3.2pt] (150.70, 78.40) rectangle (162.25, 93.20);

\path[draw=drawColor,line width= 3.2pt] (162.25, 78.40) rectangle (173.80,122.26);

\path[draw=drawColor,line width= 3.2pt] (173.80, 78.40) rectangle (185.34,132.11);

\path[draw=drawColor,line width= 3.2pt] (185.34, 78.40) rectangle (196.89,142.20);

\path[draw=drawColor,line width= 3.2pt] (196.89, 78.40) rectangle (208.44,112.28);

\path[draw=drawColor,line width= 3.2pt] (208.44, 78.40) rectangle (219.98,101.97);

\path[draw=drawColor,line width= 3.2pt] (219.98, 78.40) rectangle (231.53, 82.12);

\path[draw=drawColor,line width= 3.2pt] (231.53, 78.40) rectangle (243.07, 81.28);

\path[draw=drawColor,line width= 3.2pt] (243.07, 78.40) rectangle (254.62, 78.54);
\definecolor{drawColor}{RGB}{0,0,0}

\path[draw=drawColor,line width= 1.6pt,dash pattern=on 4pt off 4pt ,line join=round] (174.29, 72.85) -- (174.29,195.13);

\path[draw=drawColor,line width= 3.2pt,line join=round,line cap=round] ( 36.40, 72.85) rectangle (265.01,195.13);
\end{scope}
\begin{scope}
\path[clip] (  0.00,  0.00) rectangle (271.01,195.13);
\definecolor{drawColor}{RGB}{0,0,0}

\node[text=drawColor,anchor=base east,inner sep=0pt, outer sep=0pt, scale=  0.90] at ( 31.45, 75.31) {0.0};

\node[text=drawColor,anchor=base east,inner sep=0pt, outer sep=0pt, scale=  0.90] at ( 31.45,102.18) {0.2};

\node[text=drawColor,anchor=base east,inner sep=0pt, outer sep=0pt, scale=  0.90] at ( 31.45,129.06) {0.4};

\node[text=drawColor,anchor=base east,inner sep=0pt, outer sep=0pt, scale=  0.90] at ( 31.45,155.94) {0.6};

\node[text=drawColor,anchor=base east,inner sep=0pt, outer sep=0pt, scale=  0.90] at ( 31.45,182.82) {0.8};
\end{scope}
\begin{scope}
\path[clip] (  0.00,  0.00) rectangle (271.01,195.13);
\definecolor{drawColor}{gray}{0.20}

\path[draw=drawColor,line width= 3.2pt,line join=round] ( 33.65, 78.40) --
	( 36.40, 78.40);

\path[draw=drawColor,line width= 3.2pt,line join=round] ( 33.65,105.28) --
	( 36.40,105.28);

\path[draw=drawColor,line width= 3.2pt,line join=round] ( 33.65,132.16) --
	( 36.40,132.16);

\path[draw=drawColor,line width= 3.2pt,line join=round] ( 33.65,159.04) --
	( 36.40,159.04);

\path[draw=drawColor,line width= 3.2pt,line join=round] ( 33.65,185.92) --
	( 36.40,185.92);
\end{scope}
\begin{scope}
\path[clip] (  0.00,  0.00) rectangle (271.01,195.13);
\definecolor{drawColor}{gray}{0.20}

\path[draw=drawColor,line width= 3.2pt,line join=round] ( 46.79, 70.10) --
	( 46.79, 72.85);

\path[draw=drawColor,line width= 3.2pt,line join=round] (137.13, 70.10) --
	(137.13, 72.85);

\path[draw=drawColor,line width= 3.2pt,line join=round] (206.31, 70.10) --
	(206.31, 72.85);

\path[draw=drawColor,line width= 3.2pt,line join=round] (259.48, 70.10) --
	(259.48, 72.85);
\end{scope}
\begin{scope}
\path[clip] (  0.00,  0.00) rectangle (271.01,195.13);
\definecolor{drawColor}{RGB}{0,0,0}

\node[text=drawColor,anchor=base,inner sep=0pt, outer sep=0pt, scale=  0.90] at ( 46.79, 61.70) {0.001};

\node[text=drawColor,anchor=base,inner sep=0pt, outer sep=0pt, scale=  0.90] at (137.13, 61.70) {0.05};

\node[text=drawColor,anchor=base,inner sep=0pt, outer sep=0pt, scale=  0.90] at (206.31, 61.70) {1};

\node[text=drawColor,anchor=base,inner sep=0pt, outer sep=0pt, scale=  0.90] at (259.48, 61.70) {10};
\end{scope}
\begin{scope}
\path[clip] (  0.00,  0.00) rectangle (271.01,195.13);
\definecolor{drawColor}{RGB}{0,0,0}

\node[text=drawColor,anchor=base,inner sep=0pt, outer sep=0pt, scale=  0.90] at (150.70, 47.75) {Exposure to soluble MWF (mg/m\textsuperscript{3})};
\end{scope}
\begin{scope}
\path[clip] (  0.00,  0.00) rectangle (271.01,195.13);
\definecolor{drawColor}{RGB}{0,0,0}

\node[text=drawColor,rotate= 90.00,anchor=base,inner sep=0pt, outer sep=0pt, scale=  0.90] at ( 12.20,133.99) {Density};
\end{scope}
\begin{scope}
\path[clip] (  0.00,  0.00) rectangle (271.01,195.13);
\definecolor{fillColor}{RGB}{255,255,255}

\path[fill=fillColor] ( 43.99, 23.00) rectangle (257.42, 35.00);
\end{scope}
\begin{scope}
\path[clip] (  0.00,  0.00) rectangle (271.01,195.13);
\definecolor{drawColor}{RGB}{0,0,0}

\node[text=drawColor,anchor=base west,inner sep=0pt, outer sep=0pt, scale=  0.90] at ( 43.99, 30.90) {Distribution:\ \ \ };
\end{scope}
\begin{scope}
\path[clip] (  0.00,  0.00) rectangle (271.01,195.13);
\definecolor{fillColor}{RGB}{255,255,255}

\path[fill=fillColor] (105.42, 28.00) rectangle (117.42, 40.00);
\end{scope}
\begin{scope}
\path[clip] (  0.00,  0.00) rectangle (271.01,195.13);
\definecolor{drawColor}{RGB}{163,207,235}
\definecolor{fillColor}{RGB}{163,207,235}

\path[draw=drawColor,line width= 1.2pt,fill=fillColor] (106.92, 29.50) rectangle (115.92, 38.50);
\end{scope}
\begin{scope}
\path[clip] (  0.00,  0.00) rectangle (271.01,195.13);
\definecolor{fillColor}{RGB}{255,255,255}

\path[fill=fillColor] (203.85, 28.00) rectangle (215.85, 40.00);
\end{scope}
\begin{scope}
\path[clip] (  0.00,  0.00) rectangle (271.01,195.13);
\definecolor{drawColor}{RGB}{186,129,35}

\path[draw=drawColor,line width= 1.2pt] (205.35, 29.50) rectangle (214.35, 38.50);
\end{scope}
\begin{scope}
\path[clip] (  0.00,  0.00) rectangle (271.01,195.13);
\definecolor{drawColor}{RGB}{0,0,0}

\node[text=drawColor,anchor=base west,inner sep=0pt, outer sep=0pt, scale=  0.90] at (122.42, 30.90) {Post-intervention\ \ \ \ };
\end{scope}
\begin{scope}
\path[clip] (  0.00,  0.00) rectangle (271.01,195.13);
\definecolor{drawColor}{RGB}{0,0,0}

\node[text=drawColor,anchor=base west,inner sep=0pt, outer sep=0pt, scale=  0.90] at (220.85, 30.90) {Observed};
\end{scope}
\begin{scope}
\path[clip] (  0.00,  0.00) rectangle (271.01,195.13);
\definecolor{fillColor}{RGB}{255,255,255}

\path[fill=fillColor] ( 72.41,  0.00) rectangle (229.00, 12.00);
\end{scope}
\begin{scope}
\path[clip] (  0.00,  0.00) rectangle (271.01,195.13);
\definecolor{drawColor}{RGB}{0,0,0}

\node[text=drawColor,anchor=base west,inner sep=0pt, outer sep=0pt, scale=  0.90] at ( 72.41,  7.90) {Target exposure limit:};
\end{scope}
\begin{scope}
\path[clip] (  0.00,  0.00) rectangle (271.01,195.13);
\definecolor{fillColor}{RGB}{255,255,255}

\path[fill=fillColor] (164.74,  5.00) rectangle (176.74, 17.00);
\end{scope}
\begin{scope}
\path[clip] (  0.00,  0.00) rectangle (271.01,195.13);
\definecolor{drawColor}{RGB}{0,0,0}

\path[draw=drawColor,line width= 1.6pt,dash pattern=on 4pt off 4pt ,line join=round] (170.74,  5.00) -- (170.74, 17.00);
\end{scope}
\begin{scope}
\path[clip] (  0.00,  0.00) rectangle (271.01,195.13);
\definecolor{drawColor}{RGB}{0,0,0}

\node[text=drawColor,anchor=base west,inner sep=0pt, outer sep=0pt, scale=  0.90] at (181.74,  7.90) {0.25 mg/m\textsuperscript{3}};
\end{scope}

            \end{tikzpicture}
          \end{figure}

        \end{block}

      }\end{column} % End of the first column

    \begin{column}{\sepwid}\end{column} % Empty spacer column

    \begin{column}{\onecolwid} % Begin a column which is two columns wide (column 2)
      \vbox to 0.78\paperheight{

        \vspace{-0.5\baselineskip}
        \begin{figure}
          \caption{Distribution of nonzero exposure for three distinct confounder and
          exposure histories at $k = 2$ before and after applying the supportable intervention rule}
          \begin{tikzpicture}[x=2.5pt,y=2.5pt]
            % Created by tikzDevice version 0.12.3.1 on 2023-06-12 13:31:15
% !TEX encoding = UTF-8 Unicode
\definecolor{fillColor}{RGB}{255,255,255}
\path[use as bounding box,fill=fillColor,fill opacity=0.00] (0,0) rectangle (271.01,437.23);
\begin{scope}
\path[clip] (  0.00,  0.00) rectangle (271.01,437.23);
\definecolor{drawColor}{RGB}{255,255,255}
\definecolor{fillColor}{RGB}{255,255,255}

\path[draw=drawColor,line width= 0.6pt,line join=round,line cap=round,fill=fillColor] (  0.00,  0.00) rectangle (271.01,437.23);
\end{scope}
\begin{scope}
\path[clip] ( 33.90,319.44) rectangle (265.01,420.66);
\definecolor{fillColor}{RGB}{255,255,255}

\path[fill=fillColor] ( 33.90,319.44) rectangle (265.01,420.66);
\definecolor{fillColor}{RGB}{163,207,235}

\path[fill=fillColor] ( 44.40,324.04) rectangle ( 52.48,354.71);

\path[fill=fillColor] ( 52.48,324.04) rectangle ( 60.56,324.04);

\path[fill=fillColor] ( 60.56,324.04) rectangle ( 68.64,324.04);

\path[fill=fillColor] ( 68.64,324.04) rectangle ( 76.73,334.26);

\path[fill=fillColor] ( 76.73,324.04) rectangle ( 84.81,324.04);

\path[fill=fillColor] ( 84.81,324.04) rectangle ( 92.89,324.04);

\path[fill=fillColor] ( 92.89,324.04) rectangle (100.97,324.04);

\path[fill=fillColor] (100.97,324.04) rectangle (109.05,324.04);

\path[fill=fillColor] (109.05,324.04) rectangle (117.13,324.04);

\path[fill=fillColor] (117.13,324.04) rectangle (125.21,375.16);

\path[fill=fillColor] (125.21,324.04) rectangle (133.29,324.04);

\path[fill=fillColor] (133.29,324.04) rectangle (141.37,324.04);

\path[fill=fillColor] (141.37,324.04) rectangle (149.45,324.04);

\path[fill=fillColor] (149.45,324.04) rectangle (157.54,324.04);

\path[fill=fillColor] (157.54,324.04) rectangle (165.62,324.04);

\path[fill=fillColor] (165.62,324.04) rectangle (173.70,324.04);

\path[fill=fillColor] (173.70,324.04) rectangle (181.78,324.04);

\path[fill=fillColor] (181.78,324.04) rectangle (189.86,324.04);

\path[fill=fillColor] (189.86,324.04) rectangle (197.94,324.04);

\path[fill=fillColor] (197.94,324.04) rectangle (206.02,324.04);

\path[fill=fillColor] (206.02,324.04) rectangle (214.10,324.04);

\path[fill=fillColor] (214.10,324.04) rectangle (222.18,324.04);

\path[fill=fillColor] (222.18,324.04) rectangle (230.26,324.04);

\path[fill=fillColor] (230.26,324.04) rectangle (238.35,324.04);

\path[fill=fillColor] (238.35,324.04) rectangle (246.43,324.04);

\path[fill=fillColor] (246.43,324.04) rectangle (254.51,324.04);
\definecolor{drawColor}{RGB}{186,129,35}

\path[draw=drawColor,line width= 3.2pt] ( 44.40,324.04) rectangle ( 52.48,354.71);

\path[draw=drawColor,line width= 3.2pt] ( 52.48,324.04) rectangle ( 60.56,324.04);

\path[draw=drawColor,line width= 3.2pt] ( 60.56,324.04) rectangle ( 68.64,324.04);

\path[draw=drawColor,line width= 3.2pt] ( 68.64,324.04) rectangle ( 76.73,334.26);

\path[draw=drawColor,line width= 3.2pt] ( 76.73,324.04) rectangle ( 84.81,324.04);

\path[draw=drawColor,line width= 3.2pt] ( 84.81,324.04) rectangle ( 92.89,324.04);

\path[draw=drawColor,line width= 3.2pt] ( 92.89,324.04) rectangle (100.97,324.04);

\path[draw=drawColor,line width= 3.2pt] (100.97,324.04) rectangle (109.05,324.04);

\path[draw=drawColor,line width= 3.2pt] (109.05,324.04) rectangle (117.13,324.04);

\path[draw=drawColor,line width= 3.2pt] (117.13,324.04) rectangle (125.21,334.26);

\path[draw=drawColor,line width= 3.2pt] (125.21,324.04) rectangle (133.29,324.04);

\path[draw=drawColor,line width= 3.2pt] (133.29,324.04) rectangle (141.37,334.26);

\path[draw=drawColor,line width= 3.2pt] (141.37,324.04) rectangle (149.45,324.04);

\path[draw=drawColor,line width= 3.2pt] (149.45,324.04) rectangle (157.54,334.26);

\path[draw=drawColor,line width= 3.2pt] (157.54,324.04) rectangle (165.62,324.04);

\path[draw=drawColor,line width= 3.2pt] (165.62,324.04) rectangle (173.70,324.04);

\path[draw=drawColor,line width= 3.2pt] (173.70,324.04) rectangle (181.78,324.04);

\path[draw=drawColor,line width= 3.2pt] (181.78,324.04) rectangle (189.86,324.04);

\path[draw=drawColor,line width= 3.2pt] (189.86,324.04) rectangle (197.94,324.04);

\path[draw=drawColor,line width= 3.2pt] (197.94,324.04) rectangle (206.02,334.26);

\path[draw=drawColor,line width= 3.2pt] (206.02,324.04) rectangle (214.10,324.04);

\path[draw=drawColor,line width= 3.2pt] (214.10,324.04) rectangle (222.18,324.04);

\path[draw=drawColor,line width= 3.2pt] (222.18,324.04) rectangle (230.26,324.04);

\path[draw=drawColor,line width= 3.2pt] (230.26,324.04) rectangle (238.35,324.04);

\path[draw=drawColor,line width= 3.2pt] (238.35,324.04) rectangle (246.43,334.26);

\path[draw=drawColor,line width= 3.2pt] (246.43,324.04) rectangle (254.51,324.04);
\definecolor{drawColor}{RGB}{0,0,0}

\path[draw=drawColor,line width= 1.6pt,dash pattern=on 4pt off 4pt ,line join=round] (125.21,319.44) -- (125.21,420.66);

\path[draw=drawColor,line width= 3.2pt,line join=round,line cap=round] ( 33.90,319.44) rectangle (265.01,420.66);
\end{scope}
\begin{scope}
\path[clip] ( 33.90,196.14) rectangle (265.01,297.37);
\definecolor{fillColor}{RGB}{255,255,255}

\path[fill=fillColor] ( 33.90,196.14) rectangle (265.01,297.37);
\definecolor{fillColor}{RGB}{163,207,235}

\path[fill=fillColor] ( 44.40,200.74) rectangle ( 52.48,292.77);

\path[fill=fillColor] ( 52.48,200.74) rectangle ( 60.56,200.74);

\path[fill=fillColor] ( 60.56,200.74) rectangle ( 68.64,200.74);

\path[fill=fillColor] ( 68.64,200.74) rectangle ( 76.73,200.74);

\path[fill=fillColor] ( 76.73,200.74) rectangle ( 84.81,200.74);

\path[fill=fillColor] ( 84.81,200.74) rectangle ( 92.89,200.74);

\path[fill=fillColor] ( 92.89,200.74) rectangle (100.97,200.74);

\path[fill=fillColor] (100.97,200.74) rectangle (109.05,200.74);

\path[fill=fillColor] (109.05,200.74) rectangle (117.13,200.74);

\path[fill=fillColor] (117.13,200.74) rectangle (125.21,200.74);

\path[fill=fillColor] (125.21,200.74) rectangle (133.29,200.74);

\path[fill=fillColor] (133.29,200.74) rectangle (141.37,200.74);

\path[fill=fillColor] (141.37,200.74) rectangle (149.45,200.74);

\path[fill=fillColor] (149.45,200.74) rectangle (157.54,200.74);

\path[fill=fillColor] (157.54,200.74) rectangle (165.62,200.74);

\path[fill=fillColor] (165.62,200.74) rectangle (173.70,200.74);

\path[fill=fillColor] (173.70,200.74) rectangle (181.78,200.74);

\path[fill=fillColor] (181.78,200.74) rectangle (189.86,200.74);

\path[fill=fillColor] (189.86,200.74) rectangle (197.94,200.74);

\path[fill=fillColor] (197.94,200.74) rectangle (206.02,200.74);

\path[fill=fillColor] (206.02,200.74) rectangle (214.10,200.74);

\path[fill=fillColor] (214.10,200.74) rectangle (222.18,200.74);

\path[fill=fillColor] (222.18,200.74) rectangle (230.26,200.74);

\path[fill=fillColor] (230.26,200.74) rectangle (238.35,200.74);

\path[fill=fillColor] (238.35,200.74) rectangle (246.43,200.74);

\path[fill=fillColor] (246.43,200.74) rectangle (254.51,200.74);
\definecolor{drawColor}{RGB}{186,129,35}

\path[draw=drawColor,line width= 3.2pt] ( 44.40,200.74) rectangle ( 52.48,223.75);

\path[draw=drawColor,line width= 3.2pt] ( 52.48,200.74) rectangle ( 60.56,200.74);

\path[draw=drawColor,line width= 3.2pt] ( 60.56,200.74) rectangle ( 68.64,200.74);

\path[draw=drawColor,line width= 3.2pt] ( 68.64,200.74) rectangle ( 76.73,200.74);

\path[draw=drawColor,line width= 3.2pt] ( 76.73,200.74) rectangle ( 84.81,200.74);

\path[draw=drawColor,line width= 3.2pt] ( 84.81,200.74) rectangle ( 92.89,200.74);

\path[draw=drawColor,line width= 3.2pt] ( 92.89,200.74) rectangle (100.97,200.74);

\path[draw=drawColor,line width= 3.2pt] (100.97,200.74) rectangle (109.05,200.74);

\path[draw=drawColor,line width= 3.2pt] (109.05,200.74) rectangle (117.13,200.74);

\path[draw=drawColor,line width= 3.2pt] (117.13,200.74) rectangle (125.21,200.74);

\path[draw=drawColor,line width= 3.2pt] (125.21,200.74) rectangle (133.29,246.75);

\path[draw=drawColor,line width= 3.2pt] (133.29,200.74) rectangle (141.37,200.74);

\path[draw=drawColor,line width= 3.2pt] (141.37,200.74) rectangle (149.45,223.75);

\path[draw=drawColor,line width= 3.2pt] (149.45,200.74) rectangle (157.54,200.74);

\path[draw=drawColor,line width= 3.2pt] (157.54,200.74) rectangle (165.62,200.74);

\path[draw=drawColor,line width= 3.2pt] (165.62,200.74) rectangle (173.70,200.74);

\path[draw=drawColor,line width= 3.2pt] (173.70,200.74) rectangle (181.78,200.74);

\path[draw=drawColor,line width= 3.2pt] (181.78,200.74) rectangle (189.86,200.74);

\path[draw=drawColor,line width= 3.2pt] (189.86,200.74) rectangle (197.94,200.74);

\path[draw=drawColor,line width= 3.2pt] (197.94,200.74) rectangle (206.02,200.74);

\path[draw=drawColor,line width= 3.2pt] (206.02,200.74) rectangle (214.10,200.74);

\path[draw=drawColor,line width= 3.2pt] (214.10,200.74) rectangle (222.18,200.74);

\path[draw=drawColor,line width= 3.2pt] (222.18,200.74) rectangle (230.26,200.74);

\path[draw=drawColor,line width= 3.2pt] (230.26,200.74) rectangle (238.35,200.74);

\path[draw=drawColor,line width= 3.2pt] (238.35,200.74) rectangle (246.43,200.74);

\path[draw=drawColor,line width= 3.2pt] (246.43,200.74) rectangle (254.51,200.74);
\definecolor{drawColor}{RGB}{0,0,0}

\path[draw=drawColor,line width= 1.6pt,dash pattern=on 4pt off 4pt ,line join=round] (125.21,196.14) -- (125.21,297.37);

\path[draw=drawColor,line width= 3.2pt,line join=round,line cap=round] ( 33.90,196.14) rectangle (265.01,297.37);
\end{scope}
\begin{scope}
\path[clip] ( 33.90, 72.85) rectangle (265.01,174.07);
\definecolor{fillColor}{RGB}{255,255,255}

\path[fill=fillColor] ( 33.90, 72.85) rectangle (265.01,174.07);
\definecolor{fillColor}{RGB}{163,207,235}

\path[fill=fillColor] ( 44.40, 77.45) rectangle ( 52.48, 77.45);

\path[fill=fillColor] ( 52.48, 77.45) rectangle ( 60.56, 77.45);

\path[fill=fillColor] ( 60.56, 77.45) rectangle ( 68.64, 77.45);

\path[fill=fillColor] ( 68.64, 77.45) rectangle ( 76.73, 77.45);

\path[fill=fillColor] ( 76.73, 77.45) rectangle ( 84.81, 77.45);

\path[fill=fillColor] ( 84.81, 77.45) rectangle ( 92.89, 77.45);

\path[fill=fillColor] ( 92.89, 77.45) rectangle (100.97, 77.45);

\path[fill=fillColor] (100.97, 77.45) rectangle (109.05, 77.45);

\path[fill=fillColor] (109.05, 77.45) rectangle (117.13, 77.45);

\path[fill=fillColor] (117.13, 77.45) rectangle (125.21, 77.45);

\path[fill=fillColor] (125.21, 77.45) rectangle (133.29, 77.45);

\path[fill=fillColor] (133.29, 77.45) rectangle (141.37, 77.45);

\path[fill=fillColor] (141.37, 77.45) rectangle (149.45, 77.45);

\path[fill=fillColor] (149.45, 77.45) rectangle (157.54, 77.45);

\path[fill=fillColor] (157.54, 77.45) rectangle (165.62, 77.45);

\path[fill=fillColor] (165.62, 77.45) rectangle (173.70, 77.45);

\path[fill=fillColor] (173.70, 77.45) rectangle (181.78, 77.45);

\path[fill=fillColor] (181.78, 77.45) rectangle (189.86, 77.45);

\path[fill=fillColor] (189.86, 77.45) rectangle (197.94, 83.20);

\path[fill=fillColor] (197.94, 77.45) rectangle (206.02, 83.20);

\path[fill=fillColor] (206.02, 77.45) rectangle (214.10, 77.45);

\path[fill=fillColor] (214.10, 77.45) rectangle (222.18, 77.45);

\path[fill=fillColor] (222.18, 77.45) rectangle (230.26, 77.45);

\path[fill=fillColor] (230.26, 77.45) rectangle (238.35, 83.20);

\path[fill=fillColor] (238.35, 77.45) rectangle (246.43,152.22);

\path[fill=fillColor] (246.43, 77.45) rectangle (254.51, 77.45);
\definecolor{drawColor}{RGB}{186,129,35}

\path[draw=drawColor,line width= 3.2pt] ( 44.40, 77.45) rectangle ( 52.48, 77.45);

\path[draw=drawColor,line width= 3.2pt] ( 52.48, 77.45) rectangle ( 60.56, 77.45);

\path[draw=drawColor,line width= 3.2pt] ( 60.56, 77.45) rectangle ( 68.64, 77.45);

\path[draw=drawColor,line width= 3.2pt] ( 68.64, 77.45) rectangle ( 76.73, 77.45);

\path[draw=drawColor,line width= 3.2pt] ( 76.73, 77.45) rectangle ( 84.81, 77.45);

\path[draw=drawColor,line width= 3.2pt] ( 84.81, 77.45) rectangle ( 92.89, 77.45);

\path[draw=drawColor,line width= 3.2pt] ( 92.89, 77.45) rectangle (100.97, 77.45);

\path[draw=drawColor,line width= 3.2pt] (100.97, 77.45) rectangle (109.05, 77.45);

\path[draw=drawColor,line width= 3.2pt] (109.05, 77.45) rectangle (117.13, 77.45);

\path[draw=drawColor,line width= 3.2pt] (117.13, 77.45) rectangle (125.21, 77.45);

\path[draw=drawColor,line width= 3.2pt] (125.21, 77.45) rectangle (133.29, 77.45);

\path[draw=drawColor,line width= 3.2pt] (133.29, 77.45) rectangle (141.37, 77.45);

\path[draw=drawColor,line width= 3.2pt] (141.37, 77.45) rectangle (149.45, 77.45);

\path[draw=drawColor,line width= 3.2pt] (149.45, 77.45) rectangle (157.54, 77.45);

\path[draw=drawColor,line width= 3.2pt] (157.54, 77.45) rectangle (165.62, 77.45);

\path[draw=drawColor,line width= 3.2pt] (165.62, 77.45) rectangle (173.70, 77.45);

\path[draw=drawColor,line width= 3.2pt] (173.70, 77.45) rectangle (181.78, 77.45);

\path[draw=drawColor,line width= 3.2pt] (181.78, 77.45) rectangle (189.86, 77.45);

\path[draw=drawColor,line width= 3.2pt] (189.86, 77.45) rectangle (197.94, 83.20);

\path[draw=drawColor,line width= 3.2pt] (197.94, 77.45) rectangle (206.02, 83.20);

\path[draw=drawColor,line width= 3.2pt] (206.02, 77.45) rectangle (214.10, 77.45);

\path[draw=drawColor,line width= 3.2pt] (214.10, 77.45) rectangle (222.18, 77.45);

\path[draw=drawColor,line width= 3.2pt] (222.18, 77.45) rectangle (230.26, 77.45);

\path[draw=drawColor,line width= 3.2pt] (230.26, 77.45) rectangle (238.35, 83.20);

\path[draw=drawColor,line width= 3.2pt] (238.35, 77.45) rectangle (246.43,152.22);

\path[draw=drawColor,line width= 3.2pt] (246.43, 77.45) rectangle (254.51, 77.45);
\definecolor{drawColor}{RGB}{0,0,0}

\path[draw=drawColor,line width= 1.6pt,dash pattern=on 4pt off 4pt ,line join=round] (125.21, 72.85) -- (125.21,174.07);

\path[draw=drawColor,line width= 3.2pt,line join=round,line cap=round] ( 33.90, 72.85) rectangle (265.01,174.07);
\end{scope}
\begin{scope}
\path[clip] ( 33.90,174.07) rectangle (265.01,190.64);
\definecolor{drawColor}{RGB}{0,0,0}
\definecolor{fillColor}{RGB}{211,211,211}

\path[draw=drawColor,line width= 3.2pt,line join=round,line cap=round,fill=fillColor] ( 33.90,174.07) rectangle (265.01,190.64);
\definecolor{drawColor}{gray}{0.10}

\node[text=drawColor,anchor=base,inner sep=0pt, outer sep=0pt, scale=  0.88] at (149.45,179.33) {No limit (all observed exposures $>$ target limit)};
\end{scope}
\begin{scope}
\path[clip] ( 33.90,297.37) rectangle (265.01,313.94);
\definecolor{drawColor}{RGB}{0,0,0}
\definecolor{fillColor}{RGB}{211,211,211}

\path[draw=drawColor,line width= 3.2pt,line join=round,line cap=round,fill=fillColor] ( 33.90,297.37) rectangle (265.01,313.94);
\definecolor{drawColor}{gray}{0.10}

\node[text=drawColor,anchor=base,inner sep=0pt, outer sep=0pt, scale=  0.88] at (149.45,302.62) {Supported exposure limit $<$ target limit};
\end{scope}
\begin{scope}
\path[clip] ( 33.90,420.66) rectangle (265.01,437.23);
\definecolor{drawColor}{RGB}{0,0,0}
\definecolor{fillColor}{RGB}{211,211,211}

\path[draw=drawColor,line width= 3.2pt,line join=round,line cap=round,fill=fillColor] ( 33.90,420.66) rectangle (265.01,437.23);
\definecolor{drawColor}{gray}{0.10}

\node[text=drawColor,anchor=base,inner sep=0pt, outer sep=0pt, scale=  0.88] at (149.45,425.92) {Supported exposure limit $=$ target limit};
\end{scope}
\begin{scope}
\path[clip] (  0.00,  0.00) rectangle (271.01,437.23);
\definecolor{drawColor}{gray}{0.20}

\path[draw=drawColor,line width= 3.2pt,line join=round] ( 44.40, 70.10) --
	( 44.40, 72.85);

\path[draw=drawColor,line width= 3.2pt,line join=round] (109.05, 70.10) --
	(109.05, 72.85);

\path[draw=drawColor,line width= 3.2pt,line join=round] (173.70, 70.10) --
	(173.70, 72.85);

\path[draw=drawColor,line width= 3.2pt,line join=round] (238.35, 70.10) --
	(238.35, 72.85);
\end{scope}
\begin{scope}
\path[clip] (  0.00,  0.00) rectangle (271.01,437.23);
\definecolor{drawColor}{RGB}{0,0,0}

\node[text=drawColor,anchor=base,inner sep=0pt, outer sep=0pt, scale=  0.90] at ( 44.40, 61.70) {0.0};

\node[text=drawColor,anchor=base,inner sep=0pt, outer sep=0pt, scale=  0.90] at (109.05, 61.70) {0.2};

\node[text=drawColor,anchor=base,inner sep=0pt, outer sep=0pt, scale=  0.90] at (173.70, 61.70) {0.4};

\node[text=drawColor,anchor=base,inner sep=0pt, outer sep=0pt, scale=  0.90] at (238.35, 61.70) {0.6};
\end{scope}
\begin{scope}
\path[clip] (  0.00,  0.00) rectangle (271.01,437.23);
\definecolor{drawColor}{RGB}{0,0,0}

\node[text=drawColor,anchor=base east,inner sep=0pt, outer sep=0pt, scale=  0.90] at ( 28.95,320.94) {0};

\node[text=drawColor,anchor=base east,inner sep=0pt, outer sep=0pt, scale=  0.90] at ( 28.95,343.95) {10};

\node[text=drawColor,anchor=base east,inner sep=0pt, outer sep=0pt, scale=  0.90] at ( 28.95,366.95) {20};

\node[text=drawColor,anchor=base east,inner sep=0pt, outer sep=0pt, scale=  0.90] at ( 28.95,389.96) {30};

\node[text=drawColor,anchor=base east,inner sep=0pt, outer sep=0pt, scale=  0.90] at ( 28.95,412.96) {40};
\end{scope}
\begin{scope}
\path[clip] (  0.00,  0.00) rectangle (271.01,437.23);
\definecolor{drawColor}{gray}{0.20}

\path[draw=drawColor,line width= 3.2pt,line join=round] ( 31.15,324.04) --
	( 33.90,324.04);

\path[draw=drawColor,line width= 3.2pt,line join=round] ( 31.15,347.04) --
	( 33.90,347.04);

\path[draw=drawColor,line width= 3.2pt,line join=round] ( 31.15,370.05) --
	( 33.90,370.05);

\path[draw=drawColor,line width= 3.2pt,line join=round] ( 31.15,393.06) --
	( 33.90,393.06);

\path[draw=drawColor,line width= 3.2pt,line join=round] ( 31.15,416.06) --
	( 33.90,416.06);
\end{scope}
\begin{scope}
\path[clip] (  0.00,  0.00) rectangle (271.01,437.23);
\definecolor{drawColor}{RGB}{0,0,0}

\node[text=drawColor,anchor=base east,inner sep=0pt, outer sep=0pt, scale=  0.90] at ( 28.95,197.64) {0};

\node[text=drawColor,anchor=base east,inner sep=0pt, outer sep=0pt, scale=  0.90] at ( 28.95,220.65) {10};

\node[text=drawColor,anchor=base east,inner sep=0pt, outer sep=0pt, scale=  0.90] at ( 28.95,243.65) {20};

\node[text=drawColor,anchor=base east,inner sep=0pt, outer sep=0pt, scale=  0.90] at ( 28.95,266.66) {30};

\node[text=drawColor,anchor=base east,inner sep=0pt, outer sep=0pt, scale=  0.90] at ( 28.95,289.67) {40};
\end{scope}
\begin{scope}
\path[clip] (  0.00,  0.00) rectangle (271.01,437.23);
\definecolor{drawColor}{gray}{0.20}

\path[draw=drawColor,line width= 3.2pt,line join=round] ( 31.15,200.74) --
	( 33.90,200.74);

\path[draw=drawColor,line width= 3.2pt,line join=round] ( 31.15,223.75) --
	( 33.90,223.75);

\path[draw=drawColor,line width= 3.2pt,line join=round] ( 31.15,246.75) --
	( 33.90,246.75);

\path[draw=drawColor,line width= 3.2pt,line join=round] ( 31.15,269.76) --
	( 33.90,269.76);

\path[draw=drawColor,line width= 3.2pt,line join=round] ( 31.15,292.77) --
	( 33.90,292.77);
\end{scope}
\begin{scope}
\path[clip] (  0.00,  0.00) rectangle (271.01,437.23);
\definecolor{drawColor}{RGB}{0,0,0}

\node[text=drawColor,anchor=base east,inner sep=0pt, outer sep=0pt, scale=  0.90] at ( 28.95, 74.35) {0};

\node[text=drawColor,anchor=base east,inner sep=0pt, outer sep=0pt, scale=  0.90] at ( 28.95, 97.35) {10};

\node[text=drawColor,anchor=base east,inner sep=0pt, outer sep=0pt, scale=  0.90] at ( 28.95,120.36) {20};

\node[text=drawColor,anchor=base east,inner sep=0pt, outer sep=0pt, scale=  0.90] at ( 28.95,143.36) {30};

\node[text=drawColor,anchor=base east,inner sep=0pt, outer sep=0pt, scale=  0.90] at ( 28.95,166.37) {40};
\end{scope}
\begin{scope}
\path[clip] (  0.00,  0.00) rectangle (271.01,437.23);
\definecolor{drawColor}{gray}{0.20}

\path[draw=drawColor,line width= 3.2pt,line join=round] ( 31.15, 77.45) --
	( 33.90, 77.45);

\path[draw=drawColor,line width= 3.2pt,line join=round] ( 31.15,100.45) --
	( 33.90,100.45);

\path[draw=drawColor,line width= 3.2pt,line join=round] ( 31.15,123.46) --
	( 33.90,123.46);

\path[draw=drawColor,line width= 3.2pt,line join=round] ( 31.15,146.46) --
	( 33.90,146.46);

\path[draw=drawColor,line width= 3.2pt,line join=round] ( 31.15,169.47) --
	( 33.90,169.47);
\end{scope}
\begin{scope}
\path[clip] (  0.00,  0.00) rectangle (271.01,437.23);
\definecolor{drawColor}{RGB}{0,0,0}

\node[text=drawColor,anchor=base,inner sep=0pt, outer sep=0pt, scale=  0.90] at (149.45, 47.75) {Exposure to soluble MWF (mg/m\textsuperscript{3})};
\end{scope}
\begin{scope}
\path[clip] (  0.00,  0.00) rectangle (271.01,437.23);
\definecolor{drawColor}{RGB}{0,0,0}

\node[text=drawColor,rotate= 90.00,anchor=base,inner sep=0pt, outer sep=0pt, scale=  0.90] at ( 12.20,246.75) {Density};
\end{scope}
\begin{scope}
\path[clip] (  0.00,  0.00) rectangle (271.01,437.23);
\definecolor{fillColor}{RGB}{255,255,255}

\path[fill=fillColor] ( 42.74, 23.00) rectangle (256.17, 35.00);
\end{scope}
\begin{scope}
\path[clip] (  0.00,  0.00) rectangle (271.01,437.23);
\definecolor{drawColor}{RGB}{0,0,0}

\node[text=drawColor,anchor=base west,inner sep=0pt, outer sep=0pt, scale=  0.90] at ( 42.74, 30.90) {Distribution:\ \ \ };
\end{scope}
\begin{scope}
\path[clip] (  0.00,  0.00) rectangle (271.01,437.23);
\definecolor{fillColor}{RGB}{255,255,255}

\path[fill=fillColor] (104.17, 28.00) rectangle (116.17, 40.00);
\end{scope}
\begin{scope}
\path[clip] (  0.00,  0.00) rectangle (271.01,437.23);
\definecolor{drawColor}{RGB}{163,207,235}
\definecolor{fillColor}{RGB}{163,207,235}

\path[draw=drawColor,line width= 1.2pt,fill=fillColor] (105.67, 29.50) rectangle (114.67, 38.50);
\end{scope}
\begin{scope}
\path[clip] (  0.00,  0.00) rectangle (271.01,437.23);
\definecolor{fillColor}{RGB}{255,255,255}

\path[fill=fillColor] (202.60, 28.00) rectangle (214.60, 40.00);
\end{scope}
\begin{scope}
\path[clip] (  0.00,  0.00) rectangle (271.01,437.23);
\definecolor{drawColor}{RGB}{186,129,35}

\path[draw=drawColor,line width= 1.2pt] (204.10, 29.50) rectangle (213.10, 38.50);
\end{scope}
\begin{scope}
\path[clip] (  0.00,  0.00) rectangle (271.01,437.23);
\definecolor{drawColor}{RGB}{0,0,0}

\node[text=drawColor,anchor=base west,inner sep=0pt, outer sep=0pt, scale=  0.90] at (121.17, 30.90) {Post-intervention\ \ \ \ };
\end{scope}
\begin{scope}
\path[clip] (  0.00,  0.00) rectangle (271.01,437.23);
\definecolor{drawColor}{RGB}{0,0,0}

\node[text=drawColor,anchor=base west,inner sep=0pt, outer sep=0pt, scale=  0.90] at (219.60, 30.90) {Observed};
\end{scope}
\begin{scope}
\path[clip] (  0.00,  0.00) rectangle (271.01,437.23);
\definecolor{fillColor}{RGB}{255,255,255}

\path[fill=fillColor] ( 71.16,  0.00) rectangle (227.75, 12.00);
\end{scope}
\begin{scope}
\path[clip] (  0.00,  0.00) rectangle (271.01,437.23);
\definecolor{drawColor}{RGB}{0,0,0}

\node[text=drawColor,anchor=base west,inner sep=0pt, outer sep=0pt, scale=  0.90] at ( 71.16,  7.90) {Target exposure limit:};
\end{scope}
\begin{scope}
\path[clip] (  0.00,  0.00) rectangle (271.01,437.23);
\definecolor{fillColor}{RGB}{255,255,255}

\path[fill=fillColor] (163.49,  5.00) rectangle (175.49, 17.00);
\end{scope}
\begin{scope}
\path[clip] (  0.00,  0.00) rectangle (271.01,437.23);
\definecolor{drawColor}{RGB}{0,0,0}

\path[draw=drawColor,line width= 1.6pt,dash pattern=on 4pt off 4pt ,line join=round] (169.49,  5.00) -- (169.49, 17.00);
\end{scope}
\begin{scope}
\path[clip] (  0.00,  0.00) rectangle (271.01,437.23);
\definecolor{drawColor}{RGB}{0,0,0}

\node[text=drawColor,anchor=base west,inner sep=0pt, outer sep=0pt, scale=  0.90] at (180.49,  7.90) {0.25 mg/m\textsuperscript{3}};
\end{scope}

          \end{tikzpicture}
        \end{figure}

        \vspace{0.25\baselineskip}
        \begin{table}
          \caption{Demographic characteristics of the full cohort and the NHL cases}
          \small
          \begin{tabular}{p{0.3\linewidth}rlcrl}
            \toprule
                                & \multicolumn{2}{c}{Study population\phantom{0000}} &
                                & \multicolumn{2}{c}{NHL cases\phantom{00000}}                                                 \\
            \midrule
            $N$ (person-years)  & 33,134                                             & (794,733)     &  & 339  & (5,809)       \\
            Race                                                                                                               \\
            \hspace{1em}White   & 21,315                                             & (64\%)        &  & 250  & (74\%)        \\
            \hspace{1em}Black   & 6,250                                              & (19\%)        &  & 40   & (12\%)        \\
            \hspace{1em}Unknown & 5,569                                              & (17\%)        &  & 49   & (14\%)        \\
            Sex                                                                                                                \\
            \hspace{1em}Male    & 30,249                                             & (87\%)        &  & 206  & (89\%)        \\
            \hspace{1em}Female  & 4,499                                              & (13\%)        &  & 25   & (11\%)        \\
            % Plant                                                                                                                 \\
            % \hspace{1em}Plant 1    & 7,273                                              & (22\%)        &  & 70   & (21\%)        \\
            % \hspace{1em}Plant 2    & 14,251                                             & (43\%)        &  & 137  & (40\%)        \\
            % \hspace{1em}Plant 3    & 11,610                                             & (35\%)        &  & 132  & (39\%)        \\
            % Ever exposed to MWFs \\
            % \hspace{1em}Soluble & 29,010  & (88\%)    & 299  & (88\%) \\
            % \hspace{1em}Straight & 18,710  & (56\%)    & 197  & (58\%) \\
            % \hspace{1em}Synthetic & 11,824  & (36\%)  & 111  & (33\%) \\
            % Deceased by 2015         & 14,434 & (44\%)        &   & 53   & (16\%)            \\
            % % \hline
            % % Years of follow-up & 20 & (16.46, 20) &  & 13 & (6.83, 17) \\
            % Year of birth            & 1941   & (1927, 1950)  &   & 1935 & (1926, 1945)      \\
            % Year of hire             & 1967   & (1953, 1976)  &   & 1964 & (1953, 1971)      \\
            % Age at hire              & 23.6   & (20.0, 30.3)  &   & 25.3 & (20.2, 32.9)      \\
            % Year of leaving work$^*$ & 1981   & (1970, 1989)  &   & 1981 & (1971, 1989)      \\
            % Age at leaving work$^*$  & 45.2   & (31.8, 57.3)  &   & 53.0 & (36.4, 60.4)      \\
            Years at work       & 15.2                                               & (7.0, 26.6)   &  & 21.0 & (7.8, 29.9)   \\
            % Age at death$^\sharp$    & 73.4   & (64.4, 81.3)  &   & 73.0 & (66.3, 80.8)      \\
            Cumulative exposure$^\flat$
                                & 4.33                                               & (1.71, 10.69) &  & 5.43 & (2.19, 14.33) \\
            \bottomrule
            \multicolumn{6}{l}{\small Statistics shown are count (percent) or median (first and third quartiles).}             \\
            % \multicolumn{6}{l}{\small $^\sharp$ Among those who left work by 1995, when employment records ended.}                \\
            \multicolumn{6}{l}{\small $^\flat$ Among those who were ever-exposed; units in mg/m$^3\,\cdot\,$year.}             \\
            % \multicolumn{6}{l}{\small Abbreviations: NHL: non-Hodgkin lymphoma.}                                                  \\
            \bottomrule
          \end{tabular}
        \end{table}

        \vspace{1.5\baselineskip}
        \begin{block}{\normalsize Affiliations and acknowledgements \vspace*{0.2\baselineskip}}
          \vspace{-0.7\baselineskip}\small
          \begin{center}{
              \textsuperscript{1} Division of Epidemiology \& Biostatistics}
            \hspace{2em}
            \textsuperscript{2} Division of Environmental Health Sciences\\[6pt]
            Funding: CDC, NIOSH (R01OH011092) and NIOSH Training Grant (T42OH008429)
          \end{center}
        \end{block}

      }\end{column} % End of column 2

    \begin{column}{\sepwid}\end{column}

    \begin{column}{\onecolwid} % Begin a column which is two columns wide (column 2)
      \vbox to 0.78\paperheight{

        \begin{alertblock}{Main results}
          \vspace{-0.6\baselineskip}
          \begin{figure}
            \caption{Counterfactual number of cases averted under supportable intervention rules
              based on five different target exposure limits and no censoring,
              with 95\% bootstrap confidence intervals.}
            \begin{tikzpicture}[x=2.5pt,y=2.5pt]
              % Created by tikzDevice version 0.12.3.1 on 2023-06-01 11:24:54
% !TEX encoding = UTF-8 Unicode
\definecolor{fillColor}{RGB}{255,255,255}
\path[use as bounding box,fill=fillColor,fill opacity=0.00] (0,0) rectangle (271.01,162.61);
\begin{scope}
\path[clip] (  0.00,  0.00) rectangle (271.01,162.61);
\definecolor{drawColor}{RGB}{255,255,255}
\definecolor{fillColor}{RGB}{255,255,255}

\path[draw=drawColor,line width= 0.6pt,line join=round,line cap=round,fill=fillColor] ( -0.00,  0.00) rectangle (271.01,162.61);
\end{scope}
\begin{scope}
\path[clip] ( 40.89, 26.85) rectangle (265.01,162.61);
\definecolor{fillColor}{RGB}{255,255,255}

\path[fill=fillColor] ( 40.89, 26.85) rectangle (265.01,162.61);
\definecolor{drawColor}{RGB}{0,0,0}

\path[draw=drawColor,line width= 0.3pt,line join=round,line cap=round,fill=fillColor] (199.88,139.19) --
	(248.03,139.19) --
	(247.96,139.19) --
	(248.25,139.20) --
	(248.53,139.26) --
	(248.81,139.36) --
	(249.06,139.51) --
	(249.28,139.69) --
	(249.48,139.91) --
	(249.63,140.15) --
	(249.75,140.42) --
	(249.82,140.70) --
	(249.84,140.99) --
	(249.84,140.99) --
	(249.84,152.89) --
	(249.84,152.89) --
	(249.82,153.18) --
	(249.75,153.46) --
	(249.63,153.73) --
	(249.48,153.98) --
	(249.28,154.19) --
	(249.06,154.38) --
	(248.81,154.52) --
	(248.53,154.63) --
	(248.25,154.68) --
	(248.03,154.70) --
	(199.88,154.70) --
	(200.10,154.68) --
	(199.81,154.70) --
	(199.52,154.66) --
	(199.24,154.58) --
	(198.97,154.46) --
	(198.74,154.29) --
	(198.53,154.09) --
	(198.35,153.86) --
	(198.22,153.60) --
	(198.12,153.32) --
	(198.08,153.04) --
	(198.07,152.89) --
	(198.07,140.99) --
	(198.08,141.14) --
	(198.08,140.85) --
	(198.12,140.56) --
	(198.22,140.29) --
	(198.35,140.03) --
	(198.53,139.80) --
	(198.74,139.60) --
	(198.97,139.43) --
	(199.24,139.31) --
	(199.52,139.22) --
	(199.81,139.19) --
	cycle;
\end{scope}
\begin{scope}
\path[clip] ( 40.89, 26.85) rectangle (265.01,162.61);
\definecolor{drawColor}{RGB}{0,0,0}

\node[text=drawColor,anchor=base,inner sep=0pt, outer sep=0pt, scale=  0.80] at (223.95,144.19) {98 (45, 161)};
\definecolor{fillColor}{RGB}{255,255,255}

\path[draw=drawColor,line width= 0.3pt,line join=round,line cap=round,fill=fillColor] (197.88,113.08) --
	(250.03,113.08) --
	(249.96,113.08) --
	(250.25,113.09) --
	(250.53,113.15) --
	(250.81,113.25) --
	(251.06,113.40) --
	(251.28,113.58) --
	(251.48,113.80) --
	(251.63,114.05) --
	(251.75,114.31) --
	(251.81,114.60) --
	(251.84,114.89) --
	(251.84,114.89) --
	(251.84,126.78) --
	(251.84,126.78) --
	(251.81,127.07) --
	(251.75,127.36) --
	(251.63,127.62) --
	(251.48,127.87) --
	(251.28,128.09) --
	(251.06,128.27) --
	(250.81,128.42) --
	(250.53,128.52) --
	(250.25,128.58) --
	(250.03,128.59) --
	(197.88,128.59) --
	(198.10,128.58) --
	(197.81,128.59) --
	(197.52,128.55) --
	(197.24,128.47) --
	(196.97,128.35) --
	(196.74,128.18) --
	(196.53,127.98) --
	(196.35,127.75) --
	(196.22,127.49) --
	(196.12,127.22) --
	(196.08,126.93) --
	(196.07,126.78) --
	(196.07,114.89) --
	(196.08,115.03) --
	(196.08,114.74) --
	(196.12,114.45) --
	(196.22,114.18) --
	(196.35,113.92) --
	(196.53,113.69) --
	(196.74,113.49) --
	(196.97,113.32) --
	(197.24,113.20) --
	(197.52,113.12) --
	(197.81,113.08) --
	cycle;
\end{scope}
\begin{scope}
\path[clip] ( 40.89, 26.85) rectangle (265.01,162.61);
\definecolor{drawColor}{RGB}{0,0,0}

\node[text=drawColor,anchor=base,inner sep=0pt, outer sep=0pt, scale=  0.80] at (223.95,118.08) {106 (51, 173)};
\definecolor{fillColor}{RGB}{255,255,255}

\path[draw=drawColor,line width= 0.3pt,line join=round,line cap=round,fill=fillColor] (197.88, 86.97) --
	(250.03, 86.97) --
	(249.96, 86.97) --
	(250.25, 86.99) --
	(250.53, 87.04) --
	(250.81, 87.15) --
	(251.06, 87.29) --
	(251.28, 87.48) --
	(251.48, 87.69) --
	(251.63, 87.94) --
	(251.75, 88.21) --
	(251.81, 88.49) --
	(251.84, 88.78) --
	(251.84, 88.78) --
	(251.84,100.67) --
	(251.84,100.67) --
	(251.81,100.96) --
	(251.75,101.25) --
	(251.63,101.51) --
	(251.48,101.76) --
	(251.28,101.98) --
	(251.06,102.16) --
	(250.81,102.31) --
	(250.53,102.41) --
	(250.25,102.47) --
	(250.03,102.48) --
	(197.88,102.48) --
	(198.10,102.47) --
	(197.81,102.48) --
	(197.52,102.45) --
	(197.24,102.36) --
	(196.97,102.24) --
	(196.74,102.07) --
	(196.53,101.87) --
	(196.35,101.64) --
	(196.22,101.38) --
	(196.12,101.11) --
	(196.08,100.82) --
	(196.07,100.67) --
	(196.07, 88.78) --
	(196.08, 88.92) --
	(196.08, 88.63) --
	(196.12, 88.35) --
	(196.22, 88.07) --
	(196.35, 87.81) --
	(196.53, 87.58) --
	(196.74, 87.38) --
	(196.97, 87.21) --
	(197.24, 87.09) --
	(197.52, 87.01) --
	(197.81, 86.97) --
	cycle;
\end{scope}
\begin{scope}
\path[clip] ( 40.89, 26.85) rectangle (265.01,162.61);
\definecolor{drawColor}{RGB}{0,0,0}

\node[text=drawColor,anchor=base,inner sep=0pt, outer sep=0pt, scale=  0.80] at (223.95, 91.97) {112 (57, 181)};
\definecolor{fillColor}{RGB}{255,255,255}

\path[draw=drawColor,line width= 0.3pt,line join=round,line cap=round,fill=fillColor] (197.88, 60.86) --
	(250.03, 60.86) --
	(249.96, 60.87) --
	(250.25, 60.88) --
	(250.53, 60.94) --
	(250.81, 61.04) --
	(251.06, 61.18) --
	(251.28, 61.37) --
	(251.48, 61.59) --
	(251.63, 61.83) --
	(251.75, 62.10) --
	(251.81, 62.38) --
	(251.84, 62.67) --
	(251.84, 62.67) --
	(251.84, 74.57) --
	(251.84, 74.57) --
	(251.81, 74.86) --
	(251.75, 75.14) --
	(251.63, 75.41) --
	(251.48, 75.65) --
	(251.28, 75.87) --
	(251.06, 76.05) --
	(250.81, 76.20) --
	(250.53, 76.30) --
	(250.25, 76.36) --
	(250.03, 76.37) --
	(197.88, 76.37) --
	(198.10, 76.36) --
	(197.81, 76.37) --
	(197.52, 76.34) --
	(197.24, 76.26) --
	(196.97, 76.13) --
	(196.74, 75.97) --
	(196.53, 75.77) --
	(196.35, 75.53) --
	(196.22, 75.28) --
	(196.12, 75.00) --
	(196.08, 74.71) --
	(196.07, 74.57) --
	(196.07, 62.67) --
	(196.08, 62.82) --
	(196.08, 62.53) --
	(196.12, 62.24) --
	(196.22, 61.96) --
	(196.35, 61.70) --
	(196.53, 61.47) --
	(196.74, 61.27) --
	(196.97, 61.11) --
	(197.24, 60.98) --
	(197.52, 60.90) --
	(197.81, 60.87) --
	cycle;
\end{scope}
\begin{scope}
\path[clip] ( 40.89, 26.85) rectangle (265.01,162.61);
\definecolor{drawColor}{RGB}{0,0,0}

\node[text=drawColor,anchor=base,inner sep=0pt, outer sep=0pt, scale=  0.80] at (223.95, 65.86) {118 (59, 192)};
\definecolor{fillColor}{RGB}{255,255,255}

\path[draw=drawColor,line width= 0.3pt,line join=round,line cap=round,fill=fillColor] (197.88, 34.76) --
	(250.03, 34.76) --
	(249.96, 34.76) --
	(250.25, 34.77) --
	(250.53, 34.83) --
	(250.81, 34.93) --
	(251.06, 35.08) --
	(251.28, 35.26) --
	(251.48, 35.48) --
	(251.63, 35.72) --
	(251.75, 35.99) --
	(251.81, 36.27) --
	(251.84, 36.56) --
	(251.84, 36.56) --
	(251.84, 48.46) --
	(251.84, 48.46) --
	(251.81, 48.75) --
	(251.75, 49.03) --
	(251.63, 49.30) --
	(251.48, 49.54) --
	(251.28, 49.76) --
	(251.06, 49.95) --
	(250.81, 50.09) --
	(250.53, 50.19) --
	(250.25, 50.25) --
	(250.03, 50.27) --
	(197.88, 50.27) --
	(198.10, 50.25) --
	(197.81, 50.26) --
	(197.52, 50.23) --
	(197.24, 50.15) --
	(196.97, 50.02) --
	(196.74, 49.86) --
	(196.53, 49.66) --
	(196.35, 49.42) --
	(196.22, 49.17) --
	(196.12, 48.89) --
	(196.08, 48.60) --
	(196.07, 48.46) --
	(196.07, 36.56) --
	(196.08, 36.71) --
	(196.08, 36.42) --
	(196.12, 36.13) --
	(196.22, 35.85) --
	(196.35, 35.60) --
	(196.53, 35.36) --
	(196.74, 35.16) --
	(196.97, 35.00) --
	(197.24, 34.87) --
	(197.52, 34.79) --
	(197.81, 34.76) --
	cycle;
\end{scope}
\begin{scope}
\path[clip] ( 40.89, 26.85) rectangle (265.01,162.61);
\definecolor{drawColor}{RGB}{0,0,0}

\node[text=drawColor,anchor=base,inner sep=0pt, outer sep=0pt, scale=  0.80] at (223.95, 39.76) {129 (66, 214)};
\definecolor{drawColor}{RGB}{0,60,97}

\path[draw=drawColor,line width= 1.6pt,line join=round] (156.42,144.98) --
	(156.42,148.90);

\path[draw=drawColor,line width= 1.6pt,line join=round] (156.42,146.94) --
	( 84.79,146.94);

\path[draw=drawColor,line width= 1.6pt,line join=round] ( 84.79,144.98) --
	( 84.79,148.90);

\path[draw=drawColor,line width= 1.6pt,line join=round] (164.09,118.88) --
	(164.09,122.79);

\path[draw=drawColor,line width= 1.6pt,line join=round] (164.09,120.83) --
	( 88.92,120.83);

\path[draw=drawColor,line width= 1.6pt,line join=round] ( 88.92,118.88) --
	( 88.92,122.79);

\path[draw=drawColor,line width= 1.6pt,line join=round] (169.20, 92.77) --
	(169.20, 96.68);

\path[draw=drawColor,line width= 1.6pt,line join=round] (169.20, 94.73) --
	( 92.17, 94.73);

\path[draw=drawColor,line width= 1.6pt,line join=round] ( 92.17, 92.77) --
	( 92.17, 96.68);

\path[draw=drawColor,line width= 1.6pt,line join=round] (175.75, 66.66) --
	(175.75, 70.58);

\path[draw=drawColor,line width= 1.6pt,line join=round] (175.75, 68.62) --
	( 93.91, 68.62);

\path[draw=drawColor,line width= 1.6pt,line join=round] ( 93.91, 66.66) --
	( 93.91, 70.58);

\path[draw=drawColor,line width= 1.6pt,line join=round] (189.54, 40.55) --
	(189.54, 44.47);

\path[draw=drawColor,line width= 1.6pt,line join=round] (189.54, 42.51) --
	( 98.17, 42.51);

\path[draw=drawColor,line width= 1.6pt,line join=round] ( 98.17, 40.55) --
	( 98.17, 44.47);
\definecolor{fillColor}{RGB}{0,60,97}

\path[draw=drawColor,line width= 0.4pt,line join=round,line cap=round,fill=fillColor] (117.65,146.94) circle (  1.86);

\path[draw=drawColor,line width= 0.4pt,line join=round,line cap=round,fill=fillColor] (122.54,120.83) circle (  1.86);

\path[draw=drawColor,line width= 0.4pt,line join=round,line cap=round,fill=fillColor] (126.15, 94.73) circle (  1.86);

\path[draw=drawColor,line width= 0.4pt,line join=round,line cap=round,fill=fillColor] (129.92, 68.62) circle (  1.86);

\path[draw=drawColor,line width= 0.4pt,line join=round,line cap=round,fill=fillColor] (136.90, 42.51) circle (  1.86);
\definecolor{drawColor}{RGB}{0,0,0}

\path[draw=drawColor,draw opacity=0.70,line width= 0.6pt,dash pattern=on 4pt off 4pt ,line join=round] ( 57.26, 26.85) -- ( 57.26,162.61);
\definecolor{drawColor}{RGB}{0,0,0}

\path[draw=drawColor,line width= 0.9pt,line join=round,line cap=round] ( 40.89, 26.85) rectangle (265.01,162.61);
\end{scope}
\begin{scope}
\path[clip] (  0.00,  0.00) rectangle (271.01,162.61);
\definecolor{drawColor}{RGB}{0,0,0}

\node[text=drawColor,anchor=base east,inner sep=0pt, outer sep=0pt, scale=  0.90] at ( 35.94, 39.41) {0.05};

\node[text=drawColor,anchor=base east,inner sep=0pt, outer sep=0pt, scale=  0.90] at ( 35.94, 65.52) {0.25};

\node[text=drawColor,anchor=base east,inner sep=0pt, outer sep=0pt, scale=  0.90] at ( 35.94, 91.63) {0.50};

\node[text=drawColor,anchor=base east,inner sep=0pt, outer sep=0pt, scale=  0.90] at ( 35.94,117.74) {1.00};

\node[text=drawColor,anchor=base east,inner sep=0pt, outer sep=0pt, scale=  0.90] at ( 35.94,143.84) {2.00};
\end{scope}
\begin{scope}
\path[clip] (  0.00,  0.00) rectangle (271.01,162.61);
\definecolor{drawColor}{gray}{0.20}

\path[draw=drawColor,line width= 0.6pt,line join=round] ( 38.14, 42.51) --
	( 40.89, 42.51);

\path[draw=drawColor,line width= 0.6pt,line join=round] ( 38.14, 68.62) --
	( 40.89, 68.62);

\path[draw=drawColor,line width= 0.6pt,line join=round] ( 38.14, 94.73) --
	( 40.89, 94.73);

\path[draw=drawColor,line width= 0.6pt,line join=round] ( 38.14,120.83) --
	( 40.89,120.83);

\path[draw=drawColor,line width= 0.6pt,line join=round] ( 38.14,146.94) --
	( 40.89,146.94);
\end{scope}
\begin{scope}
\path[clip] (  0.00,  0.00) rectangle (271.01,162.61);
\definecolor{drawColor}{gray}{0.20}

\path[draw=drawColor,line width= 0.6pt,line join=round] ( 57.26, 24.10) --
	( 57.26, 26.85);

\path[draw=drawColor,line width= 0.6pt,line join=round] (119.00, 24.10) --
	(119.00, 26.85);

\path[draw=drawColor,line width= 0.6pt,line join=round] (180.74, 24.10) --
	(180.74, 26.85);

\path[draw=drawColor,line width= 0.6pt,line join=round] (242.48, 24.10) --
	(242.48, 26.85);
\end{scope}
\begin{scope}
\path[clip] (  0.00,  0.00) rectangle (271.01,162.61);
\definecolor{drawColor}{RGB}{0,0,0}

\node[text=drawColor,anchor=base,inner sep=0pt, outer sep=0pt, scale=  0.90] at ( 57.26, 15.70) {0};

\node[text=drawColor,anchor=base,inner sep=0pt, outer sep=0pt, scale=  0.90] at (119.00, 15.70) {100};

\node[text=drawColor,anchor=base,inner sep=0pt, outer sep=0pt, scale=  0.90] at (180.74, 15.70) {200};

\node[text=drawColor,anchor=base,inner sep=0pt, outer sep=0pt, scale=  0.90] at (242.48, 15.70) {300};
\end{scope}
\begin{scope}
\path[clip] (  0.00,  0.00) rectangle (271.01,162.61);
\definecolor{drawColor}{RGB}{0,0,0}

\node[text=drawColor,anchor=base,inner sep=0pt, outer sep=0pt, scale=  0.90] at (152.95,  1.75) {Cases averted};
\end{scope}
\begin{scope}
\path[clip] (  0.00,  0.00) rectangle (271.01,162.61);
\definecolor{drawColor}{RGB}{0,0,0}

\node[text=drawColor,rotate= 90.00,anchor=base,inner sep=0pt, outer sep=0pt, scale=  0.90] at ( 12.20, 94.73) {Target limit (mg/m$^3$)};
\end{scope}

            \end{tikzpicture}
          \end{figure}

          \begin{itemize}
            \item There would have been \textbf{502 NHL cases} if there were no censoring
            \item Stronger target exposure limits \textbf{monotonically reduced} NHL risk
            \item Setting the target exposure limit at the
                  \textbf{NIOSH recommended exposure limit}
                  0.5 mg/m\textsuperscript{3}
                  for total particulate mass derived from MWF
                  \textbf{would have averted 112 (95\% CI: 57, 181) NHL cases}
          \end{itemize}
        \end{alertblock}

        \begin{block}{Conclusions and discussion}
          \begin{itemize}
            \item \textbf{Stronger limits on exposure to soluble MWF provide stronger protections against NHL}
                  % \item Causal inference relies on assumptions\vspace*{-0.2\baselineskip}
                  %       \begin{multicols}{2}\begin{itemize}
                  %           \item Exchangeability
                  %           \item Positivity
                  %           \item Consistency
                  %           \item Correct model specification
                  %         \end{itemize}\end{multicols}
            \item During the anticipated rebound in domestic manufacturing,
                  protecting worker health should be a priority
            \item We evaluated supportable interventions with \textbf{guaranteed positivity} and therefore
                  \textbf{avoid extrapolation}
            \item We expect uniformly-enforced target exposure limits to have even
                  \textbf{stronger protective effects}
            \item The \textbf{classic parametric g-formula} estimator can also estimate
                  effects of supportable intervention rules, but
                  \textbf{requires many more parametric assumptions}
                  than ICE g-formula estimators \nocite{Taubman_2009}
          \end{itemize}
        \end{block}

        \vspace{3\baselineskip}

        \begin{block}{\normalsize References \vspace*{0.2\baselineskip}}
          \printbibliography
        \end{block}
        %     }\end{column} % End of column 2
        % \begin{column}{\sepwid}\end{column}
        % \begin{column}{\onecolwid}\vbox to 0.78\paperheight{%

        % \begin{block}{Overview of analysis steps}
        %   \begin{enumerate}\itemsep=6pt
        %     \item Fit a model for $h_{k, k - 1} = \mathbb E\left[Y_k \mid \bar l_{k - 1},
        %     \bar a_{k - 1}, Y_{k - 1} = C_k = 0\right]$.
        %     \item Predict $h^\text{int}_{k, k - 1}(a_{k - 1})$ for all possible $a_{k -1}$
        %     using the model from \mynum{1}.
        %     \item Estimate the counterfactual hazard $h^\text{int}_{K, K - 1}$
        %     $$\hat h^\text{int}_{K, K - 1} = \sum_{a_{K - 1}}
        %     \hat h^\text{int}_{k, k - 1}(a_{k - 1})
        %     \times f^\text{int}(a_{k - 1} \mid \bar l_{k - 1}, \bar a_{k - 2},
        %     Y_{k - 1} = C_{k - 1} = 0)$$
        %     \item Set $k = K - 2$. Iteratively:
        %           \vspace{2pt}\begin{enumerate}\itemsep=8pt
        %             \item Regress $\hat h^\text{int}_{K, k + 1}$ on $\bar L_k$
        %             and $\bar A_k$ among those with $Y_{k + 1} = C_{k + 1} = 0$.
        %             \item Predict $h^\text{int}_{K, k + 1} (a_k)$ for all possible $a_{k}$
        %             using the model from \mynum{a}.
        %             \item Estimate the counterfactual hazard for interval $[k, K]$
        %             \begin{align*}
        %               \hat h^\text{int}_{K, k} = \sum_{a_k}
        %               & \left(
        %                 \hat h^\text{int}_{K, k + 1} (a_k)
        %                 \times (1 - \hat h^\text{int}_{k + 1, k} (a_k)) +
        %                 \hat h^\text{int}_{k + 1, k} (a_k)
        %               \right) \\
        %               & \hspace{2em} \times f^\text{int}(a_{k} \mid
        %               \bar l_{k}, \bar a_{k - 1}, Y_{k} = C_{k} = 0)
        %             \end{align*}
        %             \item If $k > 1$, set $k \leftarrow k - 1$ and return to \mynum{a}.
        %           \end{enumerate}\vspace{2pt}
        %     \item Estimate $\mathbb E\left[ Y_K^\text{int} \right]$ using
        %     $n^{-1}\sum_i^n \hat h^\text{int}_{K, 0}$.
        %   \end{enumerate}

        % \end{block}

      }\end{column} % End of column 3

    \begin{column}{\sepwid}\end{column}

  \end{columns} % End of all the columns in the poster

\end{frame} % End of the enclosing frame

\end{document}
